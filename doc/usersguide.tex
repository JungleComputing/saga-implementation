\documentclass[a4paper,10pt]{article}

%\usepackage{graphicx}
\usepackage{url}
\usepackage{times}

\begin{document}

\title{JAVA SAGA User's Guide}

\maketitle

\section{Introduction}

This short guide describes the steps required to compile
and run an application that uses this Java SAGA implementation.
It also discusses some issues that are specific for this
Java SAGA implementation.
For information on how to write a SAGA application, the reader is referred
to the Java SAGA language binding document, the Java SAGA javadoc for the
Java language bindings for SAGA, and to the SAGA specification.
This release contains some demo programs in the \texttt{demo}
directory that will help the reader get a feel for it.

\section{Environment variables}

There is one important environment variable, used by the SAGA scripts,
and that is \texttt{JAVA\_SAGA\_LOCATION}.
This should be set and point to the root
directory of your SAGA installation. So, if you installed the SAGA
implementation in 

\noindent
{\small
\begin{verbatim}
$HOME/workspace/SagaImplementation
\end{verbatim}
}
\noindent

then you should have

\noindent
{\small
\begin{verbatim}
JAVA_SAGA_LOCATION=$HOME/workspace/SagaImplementation.
\end{verbatim}
}
\noindent

On MacOS-X, Linux, Solaris, and other Unix systems, this can be accomplished
by adding the line
\noindent
{\small
\begin{verbatim}
export JAVA_SAGA_LOCATION=$HOME/workspace/SagaImplementation
\end{verbatim}
}
\noindent
or
\noindent
{\small
\begin{verbatim}
set JAVA_SAGA_LOCATION=$HOME/workspace/SagaImplementation
\end{verbatim}
}
\noindent
for CSH users, to your .bashrc, .profile, or .cshrc file (whichever gets
executed when you log in to your system).

On Windows 2000 or Windows XP you may have installed it in, for instance,
\noindent
{\small
\begin{verbatim}
c:\Program Files\SagaImplementation
\end{verbatim}
}
\noindent
You can set the \texttt{JAVA\_SAGA\_LOCATION} variable to this path by going to the
Control Panel, System, the "Advanced" tab, Environment variables,
add it there and reboot your system.

For backwards compatibility with earlier Java SAGA releases, the scripts
also recognize the \texttt{SAGA\_LOCATION} environment variable.
However, any value of the \texttt{JAVA\_SAGA\_LOCATION} environment variable
overrides \texttt{SAGA\_LOCATION}.

\section{Compiling an application for SAGA}

The only thing that is particular to compiling an application for
Java SAGA is that the SAGA api jar-file must be on the classpath.
The SAGA API jar-file can be found
in the \texttt{lib} directory, its name starts with \texttt{saga-api-} and
ends with \texttt{.jar}. In between, there is a version number. In this
document, we'll call this file \texttt{saga-api.jar}.
The command
\noindent
{\small
\begin{verbatim}
javac -classpath $JAVA_SAGA_LOCATION/lib/saga-api.jar \
    <yourapplication>
\end{verbatim}
}
\noindent
will do the trick on Unix-like systems. On windows,
\noindent
{\small
\begin{verbatim}
javac -classpath "%JAVA_SAGA_LOCATION%"\lib\saga-api.jar
    <yourapplication>
\end{verbatim}
}
\noindent
would be the command to use (the command is split into two lines for
readability, the actual command is a single line).

Of course, it is also possible to use the \texttt{ant} program.
See for instance \texttt{build.xml} in the \texttt{demo} directory.

\section{Running a Java SAGA application}

We will now show how to run an example SAGA application.
Again, the instructions below assume that the 
\texttt{JAVA\_SAGA\_LOCATION}
environment variable is set to the location of this SAGA release.

We will use the \texttt{demo.namespace.ListTest} example here.
This application lists the directory as specified by the first
program argument with different patterns, as specified by the
remaining arguments.
The Java SAGA release contains a simple \texttt{run-saga-app} utility script.
This script creates a list of jar-files to put in the classpath (the
ones in the \texttt{lib} sub-directory of \texttt{JAVA\_SAGA\_LOCATION}),
and then calls java with the specified arguments.
On a unix-like system it can be used to run the application as follows:

\noindent
{\small
\begin{verbatim}
$ $JAVA_SAGA_LOCATION/bin/run-saga-app demo.namespace.ListTest \
    ftp://ftp.cs.vu.nl/pub/ceriel '*' '*.gz' '{L,M}*' '*tar*'
\end{verbatim}
}
\noindent

On a windows system this looks as follows:

\noindent
{\small
\begin{verbatim}
C:\DOCUME~1\Temp> "%JAVA_SAGA_LOCATION%"\bin\run-saga-app
    demo.namespace.ListTest
    ftp://ftp.cs.vu.nl/pub/ceriel "*" "*.gz" "{L,M}*" "*tar*"
\end{verbatim}
}
\noindent

Note the absence as well as presence of quoting! Also, the
command is split into multiple lines for readability.
This should be just a single line.

To run the application without \texttt{run-saga-app},
with Java 6 the following command can be used:

\noindent
{\small
\begin{verbatim}
$ java \
    -cp $JAVA_SAGA_LOCATION/lib/'*' \
    -Dsaga.location=$JAVA_SAGA_LOCATION \
    -Dlog4j.configuration=file:$JAVA_SAGA_LOCATION/log4j.properties \
    demo.namespace.ListTest \
    ftp://ftp.cs.vu.nl/pub/ceriel '*' '*.gz' '{L,M}*' '*tar*'
\end{verbatim}
}
\noindent

Note the quoting, which is significant because the shell should not expand
the '*'s. Also, an initialization file is specified for the default logger
used in the SAGA implementation.
Also, the location of Saga is passed on to Java, by means of the
\texttt{saga.location} property. The Saga Java code does not depend on
environment variables, only the scripts do.

On a windows system the command would be:

\noindent
{\small
\begin{verbatim}
C:\DOCUME~1\Temp> java
    -cp "%JAVA_SAGA_LOCATION%\lib\*"
    -Dsaga.location="%JAVA_SAGA_LOCATION%"
    -Dlog4j.configuration=file:"%JAVA_SAGA_LOCATION%"\log4j.properties
    demo.namespace.ListTest
    ftp://ftp.cs.vu.nl/pub/ceriel "*" "*.gz" "{L,M}*" "*tar*"
\end{verbatim}
}
\noindent

When using Java 1.5, the classpath must specify all jar files in the
\texttt{lib} directory. This is not shown here.

\section{Adaptors}

For most of the SAGA packages, the implementation depends on so-called
adaptors. An adaptor basically implements a SAGA package on a specific
middleware (Globus, Gridsam, sockets, \ldots).
During startup, the SAGA implementation examines which adaptors are
available (this mechanism is discussed in the next section), and loads
these. When a SAGA object is created, a corresponding set of adaptors
is instantiated, and when a
method is invoked on this object, this invocation is dynamically
dispatched to one or more of these adaptors, until one succeeds
or all adaptors fail.

The adaptors implement a specific Service Provider Interface,
corresponding to a particular interface of the Java SAGA language bindings.
The \texttt{doc} directory contains a \texttt{adaptors} subdirectory
with documentation on the currently available adaptors.

\subsection{Adaptor-specific stuff}

SAGA offers an API for accessing the Grid, and its implementation(s) use
various grid middleware to realize this. In an ideal world, SAGA would hide
all the gruesome details from the user. Unfortunately, the world is not
ideal, and some middleware may need user-defined settings.
Adaptor-specific information can be found in the corresponding adaptor
doc directory, for instance
\noindent
{\small
\begin{verbatim}
doc/adaptors/Gridsam/README.txt
\end{verbatim}
}
\noindent
on a unix system, or
\noindent
{\small
\begin{verbatim}
doc\adaptors\Gridsam\README.txt
\end{verbatim}
}
\noindent
on a windows system.

This SAGA implementation provides two mechanisms for adaptor-specific
settings.
One is by means of system properties, discussed in the
next section, which provides a mechanism for system-wide settings.
This is the preferred method for global settings.
The other is a special \texttt{preferences} SAGA context.
Such a context can be added to a specific SAGA session, and since SAGA
supports the use of multiple sessions, different adaptor instantiations
can have different settings if they have different sessions.

For example, the \texttt{demo.job.TestJob2} class has the following:
{\small
\begin{verbatim}
// Create a preferences context for JavaGAT.
// The "preferences" context is special: it is extensible.
Context context = ContextFactory.createContext("preferences");

// Make sure that javaGAT picks the wsgt4 adaptor.
context.setAttribute("ResourceBroker.adaptor.name", "wsgt4new");
context.setAttribute("wsgt4new.factory.type", "SGE");
session.addContext(context);
\end{verbatim}
}
\noindent
This sets some preferences that are specific for the JavaGAT adaptor:
it instructs JavaGAT to use its \texttt{wsgt4new} ResourceBroker adaptor
(which is the Web Services Globus4 adaptor), and instructs this
ResourceBroker adaptor to use the \texttt{SGE} scheduler.

The adaptor-specific documentation discusses all adaptor-specific
settings.

\subsection{Security}

Where needed, the user may create security contexts, and either add them
to the default session or create a special session for them.
Currently, the following security context types are recognized:

\begin{description}

\item[\texttt{ftp}]
The \texttt{ftp} context has the following default attributes:
\texttt{USERID} with value \texttt{anonymous} and \texttt{USERPASS} with
value \texttt{anonymous@localhost}.

\item[\texttt{ssh} and \texttt{sftp}]
The \texttt{ssh} and \texttt{sftp} know about the default SSH keys.

\item[\texttt{globus} and \texttt{gridftp}]
The \texttt{globus} and \texttt{gridftp} obtain the default user key
and certificate from a \texttt{.globus} sub-directory of the users' home
directory, so it sets the \texttt{USERKEY} attribute to \texttt{userkey.pem}
in that directory, and sets the \texttt{USERCERT} attribute to
\texttt{usercert.pem} in that directory.

\end{description}

Of course, the user is free to specify different values for the attributes.

\section{Properties}

The SAGA implementation recognizes a number of system properties, which
are either provided to the SAGA engine by means of a \texttt{saga.properties}
file, or by means of the \texttt{-D} option of Java.
The \texttt{saga.properties} file is searched for in the classpath and in
the current directory. If both are present, values specified in the one
in the current directory override values specified in the one in the
classpath. Also, values specified on the command line override both.

An important property is the \texttt{saga.factory} property, which
indicates the factory class that creates the factories fo the SAGA
packages.
Its value is
\noindent
{\small
\begin{verbatim}
org.ogf.saga.impl.bootstrap.MetaFactory
\end{verbatim}
}
\noindent
(see the saga.properties file in the
\texttt{JAVA\_SAGA\_LOCATION} directory), and
should not be changed, unless you have another SAGA implementation to try.

Another important property is the \texttt{saga.adaptor.path},
which tells the SAGA engine where to find the adaptors.
Its default value is \texttt{JAVA\_SAGA\_LOCATION/lib/adaptors}.
This property is interpreted is a path, which may either be specified in the
"unix" way (with '/' and ':'), or in the system-dependent way.

All properties with names ending in \texttt{.path} are subjected to
the following replacements: all occurrences of the string
\texttt{SAGA\_LOCATION} are replaced with the value of the
\texttt{JAVA\_SAGA\_LOCATION} environment variable, all occurrences
of '/' are replaced with the system-dependent separator character,
and all occurences of ':' are replaced with the system-dependent
path separator character. This allows for a system-independent way
of specifying paths in a \texttt{saga.properties} file.

Another important property, which can \emph{only} be set on the
command line, is the \texttt{java.endorsed.dirs} property. Currently,
the Gridsam adaptor only works with Java 6 if you set this property
to \texttt{\$JAVA\_SAGA\_LOCATION/lib/adaptors/GridsamAdaptor/endorsed}.
The \texttt{run-saga-app} script takes care of this, but if you don't
use this script, beware.

For the structure of an adaptor directory see the \texttt{adaptors}
sub-directory of \texttt{lib}.
In short, each adaptor has its own subdirectory, named
\texttt{<adaptorname>Adaptor},
in which a jar-file
\texttt{<adaptorname>Adaptor.jar}
exists, and which also contains
all supporting jar-files.
The manifest of
\texttt{<adaptorname>Adaptor.jar} specifies which adaptors actually
are implemented by this jar-file. For instance, 
the manifest of \texttt{GridsamAdaptor.jar} specifies:

\noindent
{\small
\begin{verbatim}
JobServiceSpi-class: org.ogf.saga.adaptors.gridsam.job.JobServiceAdaptor
\end{verbatim}
}
\noindent

which indicates that it contains a class
\noindent
{\small
\begin{verbatim}
org.ogf.saga.adaptors.gridsam.job.JobServiceAdaptor
\end{verbatim}
}
\noindent
which is an implementation of the \texttt{JobService}
Service Provicer Interface.

\section{Determining which SAGA adaptors to use}

By default, the SAGA engine tries all adaptors that it can find on the list
specified by the \texttt{saga.adaptor.path} property.
It is, however, possible to select a specific
adaptor, or to not select a specific adaptor. This is accomplished
with properties, which can be specified in a \texttt{saga.properties} file or
on the \texttt{run-saga-app} or \texttt{java} command line.
Below are some examples:

\begin{description}

\item[\texttt{StreamService.adaptor.name=socket,javagat}]
this loads both the socket and the javagat adaptor for
the StreamService SPI, but no others. Also, the adaptors will
be tried in the specified order.

\item[\texttt{StreamService.adaptor.name=!socket}]
this will load all StreamService adaptors, except for the socket
adaptor.

\end{description}

\section{Further reading}

First of all, there is the SAGA defining document:
GFD-R-P.90 (SAGA Core API) standard as defined by the
Open Grid Form (\url{http://forge.ogf.org/sf/docman/do/downloadDocument/projects.saga-core-wg/docman.root.drafts/doc13771}).

Then, there are the Java Language bindings for SAGA, which
are described in \url{http://saga.cct.lsu.edu/images/SAGA/publications/saga_core_java_binding.pdf}. This document is also included in the
\texttt{doc} directory of this release.

For the Java language bindings, detailed javadoc is available
in the \texttt{doc} directory. Just point your favorite webbrowser
to the \texttt{saga-api} sub-directory. Note: there may be a version
number in the directory name.

Last but not least, SAGA has its own website,
which can be found at \url{http://saga.cct.lsu.edu},
and which has pointers to mailing lists, other SAGA implementations,
and more documentation. Also, new releases will be announced there.

\end{document}
