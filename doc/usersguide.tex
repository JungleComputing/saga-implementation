\documentclass[a4paper,10pt]{article}

%\usepackage{graphicx}
\usepackage{url}
\usepackage{times}

\begin{document}

\title{JAVA SAGA User's Guide}

\author{Ceriel J.H. Jacobs}

\maketitle

\section{Introduction}

This short manual describes the steps required to compile
and run an application that uses this Java SAGA implementation.
The reader is referred to the
Java SAGA language binding document, the Java SAGA javadoc for the
Java language bindings for SAGA, and to the SAGA specification,
to find out how to write such an application. The current release also
contains some demo programs.

\subsection{Compiling an application for SAGA}

The only thing that is particular to compiling an application for
Java SAGA is that the SAGA api jar-file must be on the classpath.
The command
\noindent
{\small
\begin{verbatim}
javac -classpath <path to the SAGA API jar-file>
\end{verbatim}
}
\noindent
will do the trick. In this release, the SAGA API jar-file can be found
in the \texttt{lib} directory, its name is \texttt{saga-api-1.0.jar}.

Of course, it is also possible to use the \texttt{ant} program.
See for instance the \texttt{build.xml} in the \texttt{demo} directory.

\subsection{Running a Java SAGA application}

We will now show how to run an example SAGA application.
The instructions below assume that the SAGA\_LOCATION
environment variable is set to the location of this SAGA release.

We will use the \texttt{demo.namespace.ListTest} example here.
This application lists the directory as specified by the first
program argument with different patterns. The Java SAGA release
has a \texttt{run\_saga\_app} utility script, provided for
convenience. 
On a unix-like system it can be used to run the application as follows:

\noindent
{\small
\begin{verbatim}
$ $SAGA_LOCATION/bin/run_saga_app demo.namespace.ListTest \
    ftp://ftp.cs.vu.nl/pub/ceriel '*' '*.gz' '{L,M}*' '*tar*'
\end{verbatim}
}
\noindent

On a windows system this looks as follows:

\noindent
{\small
\begin{verbatim}
C:\DOCUME~1\Temp> "%SAGA_LOCATION%"\bin\run_saga_app demo.namespace.ListTest
    ftp://ftp.cs.vu.nl/pub/ceriel '*' '*.gz' '{L,M}*' '*tar*'
\end{verbatim}
}
\noindent

Note the absence as well as presence of quoting! Also, the
command is split into multiple lines for readability.
This should be just a single line.

---------- TODO ---------------

As said, the \texttt{run\_saga\_app} script is only provided for convenience.
To run the application without \texttt{run\_saga\_app}, the following command
can be used:

\noindent
{\small
\begin{verbatim}
$ java \
    -cp \
    $SAGA_LOCATION/lib/saga-demo.jar:$SAGA_LOCATION/lib/saga-api-1.0.jar: \
    -Dibis.implementation.path=$IPL_HOME/lib \
    -Dibis.server.address=localhost \
    -Dibis.pool.name=test \
    -Dlog4j.configuration=file:$IPL_HOME/log4j.properties \
    ibis.ipl.examples.Hello
\end{verbatim}
}
\noindent

In this case, we use the \texttt{ibis.implementation.path} property to supply Ibis
with the jar files of the Ibis implementations. Alternatively, they
could also all be added to the class path.


\section{Further Reading}

The Ibis web page \url{http://www.cs.vu.nl/ibis} lists all
the documentation and software available for Ibis, including papers, and
slides of presentations.

For detailed information on developing an Ibis application see the
Programmers Manual, available in the docs directory of the Ibis
distribution.

\end{document}
