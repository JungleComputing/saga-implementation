\documentclass[a4paper,10pt]{article}

%\usepackage{graphicx}
\usepackage{url}
\usepackage{times}

\begin{document}

\title{JAVA SAGA User's Guide}

\author{Ceriel J.H. Jacobs}

\maketitle

\section{Introduction}

This short manual describes the steps required to compile
and run an application that uses this Java SAGA implementation.
The reader is referred to the
Java SAGA language binding document, the Java SAGA javadoc for the
Java language bindings for SAGA, and to the SAGA specification,
to find out how to write such an application. The current release also
contains some demo programs.

\subsection{Compiling an application for SAGA}

The only thing that is particular to compiling an application for
Java SAGA is that the SAGA api jar-file must be on the classpath.
The command
\noindent
{\small
\begin{verbatim}
javac -classpath <path to the SAGA API jar-file>
\end{verbatim}
}
\noindent
will do the trick. In this release, the SAGA API jar-file can be found
in the \texttt{lib} directory, its name is \texttt{saga-api-1.0.jar}.

Of course, it is also possible to use the \texttt{ant} program.
See for instance the \texttt{build.xml} in the \texttt{demo} directory.

\subsection{Running a Java SAGA application}

We will now show how to run an example SAGA application.
The instructions below assume that the SAGA\_LOCATION
environment variable is set to the location of this SAGA release.

We will use the \texttt{demo.namespace.ListTest} example here.
This application lists the directory as specified by the first
program argument with different patterns. The Java SAGA release
has a \texttt{run\_saga\_app} utility script, provided for
convenience. 
On a unix-like system it can be used to run the application as follows:

\noindent
{\small
\begin{verbatim}
$ $SAGA_LOCATION/bin/run_saga_app demo.namespace.ListTest \
    ftp://ftp.cs.vu.nl/pub/ceriel '*' '*.gz' '{L,M}*' '*tar*'
\end{verbatim}
}
\noindent

On a windows system this looks as follows:

\noindent
{\small
\begin{verbatim}
C:\DOCUME~1\Temp> "%SAGA_LOCATION%"\bin\run_saga_app demo.namespace.ListTest
    ftp://ftp.cs.vu.nl/pub/ceriel '*' '*.gz' '{L,M}*' '*tar*'
\end{verbatim}
}
\noindent

Note the absence as well as presence of quoting! Also, the
command is split into multiple lines for readability.
This should be just a single line.

As said, the \texttt{run\_saga\_app} script is only provided for convenience.
To run the application without \texttt{run\_saga\_app},
with Java 6 the following command can be used:

\noindent
{\small
\begin{verbatim}
$ java \
    -cp $SAGA_LOCATION/lib/'*' \
    -Dlog4j.configuration=file:$SAGA_LOCATION/log4j.properties \
    demo.namespace.ListTest \
    ftp://ftp.cs.vu.nl/pub/ceriel '*' '*.gz' '{L,M}*' '*tar*'
\end{verbatim}
}
\noindent

Note the quoting, which is significant because the shell should not expand
the '*'s.

On a windows system the command would be:

\noindent
{\small
\begin{verbatim}
C:\DOCUME~1\Temp> java
    -cp "%SAGA_LOCATION%"\lib\'*'
    -Dlog4j.configuration=file:"%SAGA_LOCATION%"\log4j.properties \
    ftp://ftp.cs.vu.nl/pub/ceriel '*' '*.gz' '{L,M}*' '*tar*'
\end{verbatim}
}
\noindent

When using Java 1.5, the classpath must specify all jar files in the lib
directory.

\subsubsection{Properties}

The SAGA implementation recognizes a number of system properties, which
are either provided to the SAGA engine by means of a \texttt{saga.properties}
file, or by means of the \texttt{-D} option of Java.
The \texttt{saga.properties} file is searched for in the classpath and in
the current directory. If both are present, values specified in the one
in the current directory override values specified in the one in the
classpath. Also, values specified on the command line override both.

An important property is the \texttt{saga.factory} property, which
indicates the factory class that creates the factories fo the SAGA
packages. Its value is \texttt{org.ogf.saga.impl.bootstrap.MetaFactory}
(see the saga.properties file in the SAGA\_LOCATION directory), and
should not be changed, unless you have another SAGA implementation to try.

Another important property is the \texttt{saga.adaptor.path},
which tells the SAGA engine where to find the adaptors.
Its default value is \texttt{SAGA\_LOCATION/lib/adaptors}, 
This is a path, which may either be specified in the "unix" way
(with '/' and ':'), or in the system-dependent way.





\end{document}
